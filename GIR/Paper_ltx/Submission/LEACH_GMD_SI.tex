%% Copernicus Publications Manuscript Preparation Template for LaTeX Submissions
%% ---------------------------------
%% This template should be used for copernicus.cls
%% The class file and some style files are bundled in the Copernicus Latex Package, which can be downloaded from the different journal webpages.
%% For further assistance please contact Copernicus Publications at: production@copernicus.org
%% https://publications.copernicus.org/for_authors/manuscript_preparation.html


%% Please use the following documentclass and journal abbreviations for discussion papers and final revised papers.

%% 2-column papers and discussion papers
\documentclass[gmd, manuscript]{copernicus}



%% Journal abbreviations (please use the same for discussion papers and final revised papers)


% Advances in Geosciences (adgeo)
% Advances in Radio Science (ars)
% Advances in Science and Research (asr)
% Advances in Statistical Climatology, Meteorology and Oceanography (ascmo)
% Annales Geophysicae (angeo)
% Archives Animal Breeding (aab)
% ASTRA Proceedings (ap)
% Atmospheric Chemistry and Physics (acp)
% Atmospheric Measurement Techniques (amt)
% Biogeosciences (bg)
% Climate of the Past (cp)
% DEUQUA Special Publications (deuquasp)
% Drinking Water Engineering and Science (dwes)
% Earth Surface Dynamics (esurf)
% Earth System Dynamics (esd)
% Earth System Science Data (essd)
% E&G Quaternary Science Journal (egqsj)
% European Journal of Mineralogy (ejm)
% Fossil Record (fr)
% Geochronology (gchron)
% Geographica Helvetica (gh)
% Geoscience Communication (gc)
% Geoscientific Instrumentation, Methods and Data Systems (gi)
% Geoscientific Model Development (gmd)
% History of Geo- and Space Sciences (hgss)
% Hydrology and Earth System Sciences (hess)
% Journal of Micropalaeontology (jm)
% Journal of Sensors and Sensor Systems (jsss)
% Magnetic Resonance (mr)
% Mechanical Sciences (ms)
% Natural Hazards and Earth System Sciences (nhess)
% Nonlinear Processes in Geophysics (npg)
% Ocean Science (os)
% Primate Biology (pb)
% Proceedings of the International Association of Hydrological Sciences (piahs)
% Scientific Drilling (sd)
% SOIL (soil)
% Solid Earth (se)
% The Cryosphere (tc)
% Weather and Climate Dynamics (wcd)
% Web Ecology (we)
% Wind Energy Science (wes)


%% \usepackage commands included in the copernicus.cls:
%\usepackage[german, english]{babel}
%\usepackage{tabularx}
%\usepackage{cancel}
%\usepackage{multirow}
%\usepackage{supertabular}
%\usepackage{algorithmic}
%\usepackage{algorithm}
%\usepackage{amsthm}
%\usepackage{float}
%\usepackage{subfig}
%\usepackage{rotating}


\begin{document}

\title{Supplementary Information for \emph{GIR: a generalised impulse-response model for climate uncertainty and future scenario exploration}}


% \Author[affil]{given_name}{surname}

\Author[1]{Nicholas J.}{Leach}
\Author[2,3]{Zebedee}{Nicholls}
\Author[1]{Stuart}{Jenkins}
\Author[4]{Christopher J.}{Smith}
\Author[1]{John}{Lynch}
\Author[1]{Michelle}{Cain}
\Author[1]{Bill}{Wu}
\Author[5]{Junichi}{Tsutsui}
\Author[1,6]{Myles R.}{Allen}

\affil[1]{Department of Physics, Atmospheric Oceanic and Planetary Physics, University of Oxford, United Kingdom.}
\affil[2]{Australian--German Climate and Energy College, University of Melbourne, Australia.}
\affil[3]{School of Earth Sciences, University of Melbourne, Australia.}
\affil[4]{School of Earth and Environment, University of Leeds, Leeds, UK.}
\affil[5]{Environmental Science Laboratory, Central Research Institute of Electric Power Industry, Abiko-shi, Japan.}
\affil[6]{Environmental Change Institute, University of Oxford, Oxford, UK.}

%% The [] brackets identify the author with the corresponding affiliation. 1, 2, 3, etc. should be inserted.

%% If an author is deceased, please add a further affiliation and mark the respective author name(s) with a dagger, e.g. "\Author[2,$\dag$]{Anton}{Aman}" with the affiliations "\affil[2]{University of ...}" and "\affil[$\dag$]{deceased, 1 July 2019}"


\correspondence{Nicholas J. Leach (nicholas.leach@stx.ox.ac.uk)}

\runningtitle{TEXT}

\runningauthor{TEXT}





\received{}
\pubdiscuss{} %% only important for two-stage journals
\revised{}
\accepted{}
\published{}

%% These dates will be inserted by Copernicus Publications during the typesetting process.


\firstpage{1}

\maketitle

\copyrightstatement{@ Author(s) 2019. This work is distributed under
the Creative Commons Attribution 4.0 License.}

\begin{table}[t]
    \caption{Units used in GIR when the default parameter set is used for each gas or aerosol species. Default forcing unit for all species is Wm$^{-2}$.} \label{tab:units}
    \begin{tabular}{l r r r r r r r r r r}
    \tophline
    Variable & CO$_2$ & CH$_4$ & N$_2$O & SOx & NOx & BC & OC & NH3 & VOC & All other WMGHGs\\
    \middlehline
    Emissions & GtC & MtCH$_4$ & MtN$_2$O-N$_2$ & MtSO$-2$ & MtN & MtC & MtC & Mt & Mt & Mt  \\
    Concentrations & ppm & ppb & ppb & - & - & - & - & - & - & ppb \\
    \bottomhline
    \end{tabular}
    \belowtable{} % Table Footnotes
\end{table}
\clearpage
\begin{figure}[t]
    \includegraphics[width=\textwidth]{"../../Final_figures/iIRF100_numeric_analytic".pdf}
    \caption{Numerical solution for $\alpha$ in FaIR v1.0 and v1.3 versus analytic solution in GIR. This highlights the reason for the lowered $r_0$ parameter in GIR compared to FaIR v1.3 \citep{Smith2017} or v1.0 \citep{Millar2016}, as we see that for pre-industrial $\alpha$ values (0.12 in v1.0 or 0.16 in v1.3), the analytic solution requires a lower iIRF$_{100}$ than the numeric. Despite the marginal difference in shape, we do not find GIR has reduced ability in reproducing historical concentration when compared with FaIR v1.0 or v1.3.}
\end{figure}
\clearpage
\begin{figure}[t]
    \includegraphics[width=\textwidth]{"../../Final_figures/RCP concetration driven residual emissions".pdf}
    \caption{Differences between historical diagnosed emissions in GIR and the RCP database emissions for CH$_4$ and N$_2$O. This displays high similarity to the imposed natural emissions in FaIR v1.3 (Figure 2 from \cite{Smith2017}), demonstrating that CH$_4$ and N$_2$O cycles in GIR and FaIR v1.3 are not systematically different.}
\end{figure}
\clearpage
\begin{figure}[t]
    \includegraphics[width=\textwidth]{"../../Final_figures/Rigby_tuned_other_gases_plus_RCP".pdf}
    \caption{Best-estimate annual emissions from a more complex atmospheric model inversion \cite{Rigby2014}, GIR inverse emissions and RCP database emissions. Open black circles show inverse emissions from a 12-box model \cite{Cunnold1994}; solid green lines show inverse emissions from GIR with tuned parameters and solid pink lines show emissions from the RCP database. Inset text shows the tuned species lifetime.}
\end{figure}
\clearpage
\begin{figure}[t]
    \includegraphics[width=\textwidth]{"../../Final_figures/RF deviation versus Etminan over RCPs".pdf}
    \caption{Corresponding deviations of FaIR v2.0 from \cite{Etminan2016} formulae for the RCPs, largely caused by inter-gas interaction terms. These are plotted as \% deviations against gas concentration (so absolute deviations are considerably lower at low concentrations than high concentrations).}
\end{figure}
\clearpage
\section{RCP simulations}
This section includes plots of GIR response to the Representative Concentration Pathways; as their use has been widespread in model intercomparisons. Since the RCPs themselves are \emph{concentration} pathways, here we focus on the results of running GIR with concentrations from the RCP database, as was done in the GCMs in CMIP5. Figure \ref{fig:RCPconcs} shows concentrations resulting from driving GIR with the RCP emissions, demonstrating the incompatibilities between RCP concentrations and emissions due to the lack of integration as was planned in \cite{Moss2010}. Figure \ref{fig:RCPemms} shows diagnosed emissions that are compatible with the RCP concentration series in GIR, run with default parameters plus uncertainties as described above. Here we see the large discrepancies between GIR compatible emissions and the RCP database emissions for CH$_4$ and N$_2$O, while GIR diagnosed emissions agree well with bottom-up emission estimates to the present-day as expected, since GIR is tuned against a very similar historical concentration series; though here we run GIR with the CH$_4$ $r_0$ parameter tuned to the Global Methane Budget \citep{Saunois}, hence the slight discrepancy between GIR and PRIMAP-histTP.
\begin{figure}[t]
    \includegraphics[width=\textwidth]{"../../Final_figures/RCP concentration driven emissions".pdf}
    \caption{Diagnosed emissions corresponding to the RCPs. Solid lines show best-estimate GIR diagnosed inverse emissions, with associated 5-95\% plumes. Dotted lines show emissions directly from the RCP database. Unfilled black circles show bottom-up emission estimates from GCP and PRIMAP-histTP \citep{Quere2018,Gutschow2016}, smoothed with a 5-yearly running mean.}
    \label{fig:RCPemms}
\end{figure}
\clearpage
Figure \ref{fig:RCPRF} shows the corresponding radiative forcings compared to those from the RCP database. The updated simple RF formulae described in \cite{Etminan2016} increase the RF of CH$_4$ and CO$_2$, while marginally decreasing the N$_2$O RF.
\begin{figure}[t]
    \includegraphics[width=\textwidth]{"../../Final_figures/RCP concentration driven RF".pdf}
    \caption{Radiative forcing as computed from the RCP database concentrations using GIR, and corresponding forcings diagnosed by MAGICC from the database.}
    \label{fig:RCPRF}
\end{figure}
\clearpage
Figure \ref{fig:RCPtemps} shows historical and future global mean surface temperature ranges (GMST) under the RCP scenarios diagnosed by GIR, alongside historical observed data and future projections from CMIP5. We see that the GIR diagnosed temperatures closely resemble the observed GMST series, but are lower than the CMIP5 ranges. This is due to the default climate response parameter selection in GIR as described above and is discussed in \cite{Richardson2016}.
\begin{figure}[t]
    \includegraphics[width=\textwidth]{"../../Final_figures/RCP concentration driven temperatures".pdf}
    \caption{RCP temperature anomalies computed by GIR, and ranges from the CMIP5 ensemble. Solid lines and plumes show best-estimate and 5-95\% range temperature anomaly simulated in GIR. Unfilled black circles show observed GMST as the mean of 4 temperature datasets \citep{Vose2012,Cowtan2014,Lenssen2019,Morice2011}, smoothed with a 5-yearly running mean. Error bars in 2090 show CMIP5 projected 2081:2100 mean temperature anomalies \citep{Collins2013}.}
    \label{fig:RCPtemps}
\end{figure}
\clearpage
\begin{figure}[t]
    \includegraphics[width=\textwidth]{"../../Final_figures/RCP emission driven concentrations".pdf}
    \caption{Simulated concentrations when RCP database emissions and other forcings drive GIR, compared to original RCPs. This highlights the historical and future problems of compatibility between the RCP emission and concentration timeseries, especially for N$_2$O. This issue is discussed more fully in the ``Specification of natural emissions'' section in the main text.}
    \label{fig:RCPconcs}
\end{figure}
\clearpage

%% The following commands are for the statements about the availability of data sets and/or software code corresponding to the manuscript.
%% It is strongly recommended to make use of these sections in case data sets and/or software code have been part of your research the article is based on.

% \codeavailability{TEXT} %% use this section when having only software code available


% \dataavailability{TEXT} %% use this section when having only data sets available


\codedataavailability{The model and code used to produce the figures is publicly available at \url{https://github.com/njleach/GIR}, and will be cleaned up and release ready prior to acceptance. All data used in this study is publicly available at the relevant cited sources.} %% use this section when having data sets and software code available


% \sampleavailability{TEXT} %% use this section when having geoscientific samples available


% \videosupplement{TEXT} %% use this section when having video supplements available


% \appendix
% \section{}    %% Appendix A

% \subsection{}     %% Appendix A1, A2, etc.


\noappendix       %% use this to mark the end of the appendix section

%% Regarding figures and tables in appendices, the following two options are possible depending on your general handling of figures and tables in the manuscript environment:

%% Option 1: If you sorted all figures and tables into the sections of the text, please also sort the appendix figures and appendix tables into the respective appendix sections.
%% They will be correctly named automatically.

%% Option 2: If you put all figures after the reference list, please insert appendix tables and figures after the normal tables and figures.
%% To rename them correctly to A1, A2, etc., please add the following commands in front of them:

\appendixfigures  %% needs to be added in front of appendix figures

\appendixtables   %% needs to be added in front of appendix tables

%% Please add \clearpage between each table and/or figure. Further guidelines on figures and tables can be found below.



% \authorcontribution{NJL, SJ and MRA conceived the study. NJL and SJ wrote the model code, and BW helped tune model parameters. JT provided CMIP6 response parameters. JL and MC advised on model uses and tested the model. NJL, CJS, ZN, JL and MRA wrote the manuscript.} %% this section is mandatory

% \competinginterests{We declare that we have no competing interests.} %% this section is mandatory even if you declare that no competing interests are present

% \disclaimer{TEXT} %% optional section

% \begin{acknowledgements}
% We acknowledge the World Climate Research Programme, which, through its Working Group on Coupled Modelling, coordinated and promoted both CMIP5 and CMIP6.
% \end{acknowledgements}




%% REFERENCES

%% The reference list is compiled as follows:

% \begin{thebibliography}{}

% \bibitem[AUTHOR(YEAR)]{LABEL1}
% REFERENCE 1

% \bibitem[AUTHOR(YEAR)]{LABEL2}
% REFERENCE 2

% \end{thebibliography}

%% Since the Copernicus LaTeX package includes the BibTeX style file copernicus.bst,
%% authors experienced with BibTeX only have to include the following two lines:
%%
\bibliographystyle{copernicus}
\bibliography{GIR_bib.bib}
%%
%% URLs and DOIs can be entered in your BibTeX file as:
%%
%% URL = {http://www.xyz.org/~jones/idx_g.htm}
%% DOI = {10.5194/xyz}


%% LITERATURE CITATIONS
%%
%% command                        & example result
%% \citet{jones90}|               & Jones et al. (1990)
%% \citep{jones90}|               & (Jones et al., 1990)
%% \citep{jones90,jones93}|       & (Jones et al., 1990, 1993)
%% \citep[p.~32]{jones90}|        & (Jones et al., 1990, p.~32)
%% \citep[e.g.,][]{jones90}|      & (e.g., Jones et al., 1990)
%% \citep[e.g.,][p.~32]{jones90}| & (e.g., Jones et al., 1990, p.~32)
%% \citeauthor{jones90}|          & Jones et al.
%% \citeyear{jones90}|            & 1990



%% FIGURES

%% When figures and tables are placed at the end of the MS (article in one-column style), please add \clearpage
%% between bibliography and first table and/or figure as well as between each table and/or figure.


%% ONE-COLUMN FIGURES

%%f
%\begin{figure}[t]
%\includegraphics[width=8.3cm]{FILE NAME}
%\caption{TEXT}
%\end{figure}
%
%%% TWO-COLUMN FIGURES
%
%%f
%\begin{figure*}[t]
%\includegraphics[width=12cm]{FILE NAME}
%\caption{TEXT}
%\end{figure*}
%
%
%%% TABLES
%%%
%%% The different columns must be seperated with a & command and should
%%% end with \\ to identify the column brake.
%
%%% ONE-COLUMN TABLE
%
%%t
%\begin{table}[t]
%\caption{TEXT}
%\begin{tabular}{column = lcr}
%\tophline
%
%\middlehline
%
%\bottomhline
%\end{tabular}
%\belowtable{} % Table Footnotes
%\end{table}
%
%%% TWO-COLUMN TABLE
%
%%t
%\begin{table*}[t]
%\caption{TEXT}
%\begin{tabular}{column = lcr}
%\tophline
%
%\middlehline
%
%\bottomhline
%\end{tabular}
%\belowtable{} % Table Footnotes
%\end{table*}
%
%%% LANDSCAPE TABLE
%
%%t
%\begin{sidewaystable*}[t]
%\caption{TEXT}
%\begin{tabular}{column = lcr}
%\tophline
%
%\middlehline
%
%\bottomhline
%\end{tabular}
%\belowtable{} % Table Footnotes
%\end{sidewaystable*}
%
%
%%% MATHEMATICAL EXPRESSIONS
%
%%% All papers typeset by Copernicus Publications follow the math typesetting regulations
%%% given by the IUPAC Green Book (IUPAC: Quantities, Units and Symbols in Physical Chemistry,
%%% 2nd Edn., Blackwell Science, available at: http://old.iupac.org/publications/books/gbook/green_book_2ed.pdf, 1993).
%%%
%%% Physical quantities/variables are typeset in italic font (t for time, T for Temperature)
%%% Indices which are not defined are typeset in italic font (x, y, z, a, b, c)
%%% Items/objects which are defined are typeset in roman font (Car A, Car B)
%%% Descriptions/specifications which are defined by itself are typeset in roman font (abs, rel, ref, tot, net, ice)
%%% Abbreviations from 2 letters are typeset in roman font (RH, LAI)
%%% Vectors are identified in bold italic font using \vec{x}
%%% Matrices are identified in bold roman font
%%% Multiplication signs are typeset using the LaTeX commands \times (for vector products, grids, and exponential notations) or \cdot
%%% The character * should not be applied as mutliplication sign
%
%
%%% EQUATIONS
%
%%% Single-row equation
%
%\begin{equation}
%
%\end{equation}
%
%%% Multiline equation
%
%\begin{align}
%& 3 + 5 = 8\\
%& 3 + 5 = 8\\
%& 3 + 5 = 8
%\end{align}
%
%
%%% MATRICES
%
%\begin{matrix}
%x & y & z\\
%x & y & z\\
%x & y & z\\
%\end{matrix}
%
%
%%% ALGORITHM
%
%\begin{algorithm}
%\caption{...}
%\label{a1}
%\begin{algorithmic}
%...
%\end{algorithmic}
%\end{algorithm}
%
%
%%% CHEMICAL FORMULAS AND REACTIONS
%
%%% For formulas embedded in the text, please use \chem{}
%
%%% The reaction environment creates labels including the letter R, i.e. (R1), (R2), etc.
%
%\begin{reaction}
%%% \rightarrow should be used for normal (one-way) chemical reactions
%%% \rightleftharpoons should be used for equilibria
%%% \leftrightarrow should be used for resonance structures
%\end{reaction}
%
%
%%% PHYSICAL UNITS
%%%
%%% Please use \unit{} and apply the exponential notation


\end{document}
